\documentclass{article}
\usepackage{algorithm}
\usepackage{algpseudocode}
\usepackage{amsmath,amssymb,amsthm}
\usepackage{graphicx}
\usepackage[margin=1in]{geometry}
\usepackage{fancyhdr}
\usepackage{longtable}
\usepackage{lipsum}
\setlength{\parindent}{0pt}
\setlength{\parskip}{5pt plus 1pt}
\setlength{\headheight}{13.6pt}
\newcommand\question[2]{\vspace{.25in}\hrule\textbf{#1: #2}\hrule\vspace{.10in}}
\renewcommand\part[1]{\vspace{.10in}\textbf{(#1)}}
\newcommand\algo{\vspace{.10in}\textbf{Algorithm: }}
\newcommand\correctness{\vspace{.10in}\textbf{Correctness: }}
\newcommand\runtime{\vspace{.10in}\textbf{Running time: }}
\newcommand\pseudoCode{\vspace{.10in}\textbf{PseudoCode: }}
\newcommand*{\perm}[2]{{}^{#1}\!P_{#2}}
\newcommand*{\comb}[2]{{}^{#1}\!C_{#2}}
%\pagestyle{fancyplain}
%\lhead{\textbf{\NAME\ (\UID)}}
%\chead{\textbf{Hw\HWNUM}}
%\rhead{CS 6150, \today}
\title{CS6150 - Homework/Assignment-3}
\author{Arnab Das(u1014840)}
\usepackage[utf8]{inputenc}
\begin{document}
  \pagenumbering{gobble}
  \maketitle
  \newpage
  \pagenumbering{arabic}
  \newcommand\NAME{ARNAB DAS}
  \newcommand\UID{uxxxxxxx}
  \newcommand\HWNUM{3}

  \question{1}{easy relatives of 3-SAT}
	\part{a} Given a caluse of a 2-CNF formula $x_{1} \vee x_{2}$. Goal is to prove that this clause is logically equivalent to the clause $x_{1} \Rightarrow x_{2}$.  The implication clause $x_{1} \Rightarrow x_{2}$ means, if $x_{1}$ is true, it implies $x_{2}$, else if $x_{1}$ is false then the implication doesn't fires and the result is true. Thus, when $x_{1}$ is false(or $\bar {x_{1}}$ is true), the clause is always true regardless of $x_{2}$. When $x_{1}$ is true, it implies the clause is determined by the result of $x_{2}$, that is if $x_{2}$ is true, then the result is true, and if $x_{2}$ is false, the result is false. Thus, if we make the truth table of $x_{1} \Rightarrow x_{2}$, the table will look as below: \newline
\begin{table}[ht]
  \caption{Truth Table of $x_{1} \Rightarrow x_{2}$}
  \centering
  \begin{tabular}{c c c }
  \hline\hline
  $x_{1}$ & $x_{2}$ & $x_{1} \Rightarrow x_{2}$ \\[0.5ex]
  \hline
  0 & 0 & 1 \\
  0 & 1 & 1 \\
  1 & 0 & 0 \\
  1 & 1 & 1 \\ [0.5ex]
  \end{tabular}
  \label{table:nonlin}
  \end{table}	  

Below is the truth table of $\bar{x_{1}} \vee x_{2}$ \newline
\begin{table}[ht]
  \caption{Truth Table of $\bar{x_{1}} \vee x_{2}$}
  \centering
  \begin{tabular}{c c c }
  \hline\hline
  $x_{1}$ & $x_{2}$ & $\bar{x_{1}} \vee x_{2}$ \\[0.5ex]
  \hline
  0 & 0 & 1 \\
  0 & 1 & 1 \\
  1 & 0 & 0 \\
  1 & 1 & 1 \\ [0.5ex]
  \end{tabular}
  \label{table:nonlin}
  \end{table}	 

  The truth tables of table-1 and table-2 are exactly the same thus implying that the clause $\bar{x_{1}} \vee x_{2}$ is equivalent to $x_{1} \Rightarrow x_{2}$. \newline

  \part{b} Suppose we have a 2-CNF formula, comprising of m-clauses , such that each clause is of the form $x_{i} \vee x_{j}$, which means it is equivalent to (from our previous proof) both $\bar{x_{1}} \Rightarrow x_{2}$ and $\bar{x_{2}} \Rightarrow x_{1}$

\end{document} 
